\documentclass[a4paper, 11pt]{article}
\usepackage{amsmath, amssymb, graphicx}
\title{Appleby Quantum Gravity Framework}
\author{Ethan G. Appleby}
\date{\today}

\begin{document}

\maketitle

\begin{abstract}
A quantum gravity framework that uses a GR interpretation over hyper luminal quantum effects as sub luminal.
\end{abstract}

\section{Model Formulation}

\subsection{Invariant Rate of Entropy is corralation range determined via c_eff}

\begin{equation}
E = \sqrt{\frac{c^5 \cdot \hbar}{4\pi G}} \cdot \sqrt{\ln(N)}
\end{equation}

where

\begin{equation}
\ln(N) = \frac{A \cdot c^3}{4G \cdot \hbar}
\end{equation}

So

\begin{equation}
E = \sqrt{\frac{c^5 \cdot \hbar}{4\pi G}} \cdot \sqrt{\frac{A \cdot c^3}{4G \cdot \hbar}}
\end{equation}

Simplifying further:

\begin{equation}
E = \sqrt{\frac{c^8 \cdot A}{16\pi G^2}}
\end{equation}

Using

\begin{equation}
A = 16\pi \cdot \left(\frac{GM}{c^2}\right)^2
\end{equation}

Substituting \( A \) into the energy equation:

\begin{equation}
E = \sqrt{\frac{c^8 \cdot 16\pi \cdot \left(\frac{G^2 \cdot M^2}{c^4}\right)}{16\pi G^2}}
\end{equation}

Simplifying further:

\begin{equation}
E = \sqrt{\frac{c^8 \cdot G^2 \cdot M^2}{c^4 \cdot G^2}} = \sqrt{c^4 \cdot M^2} = c^2 \cdot M
\end{equation}

Thus, re-arranging,

\begin{equation}
E = mc^2
\end{equation}

\subsection{Area of a Horizon}
\begin{equation}
\frac{A}{\sqrt{1 - \frac{v^2}{c^2}}} = \frac{4G\hbar \ln(N_{\text{rest}})}{c^3}
\end{equation}

Where \( A_{\text{Planck}} \) is the Planck area:

\begin{equation}
\frac{A}{\sqrt{1 - \frac{v^2}{c^2}}} = A_{\text{Planck}}
\end{equation}

So,

\begin{equation}
\sqrt{1 - \frac{v^2}{c^2}} = \frac{A_{\text{Planck}}}{A}
\end{equation}

and

\begin{equation}
\frac{v^2}{c^2} = 1 - \left(\frac{A_{\text{Planck}}}{A}\right)^2
\end{equation}

Thus,

\begin{equation}
v = c \cdot \sqrt{1 - \left(\frac{A_{\text{Planck}}}{A}\right)^2}
\end{equation}

\subsection{Black Hole Mergers with Negentropy}

The area after a merger:

\begin{equation}
A_1 + A_2 = A_3 \text{ (or less } A_3)
\end{equation}

\subsection{Curvature Tensor}
\begin{equation}
G_{\mu\nu} = \rho \cdot \left[ \frac{c^2 \cdot H_{\text{obs}}^2 \cdot \Delta_H}{H^3} - \frac{8 \pi G}{c^4} + \left(4 \pi H_{\text{universe}}^2\right)^2 \right]
\end{equation}

\subsection{Effective Speed of Light}
\begin{equation}
c_{\text{eff}} = c \cdot \frac{H_{\text{obs}}}{H}
\end{equation}

\subsection{Gravitational Potential Energy}
\begin{equation}
U_{\text{Appleby}} = -\frac{c_{\text{eff}}^2}{8 \pi G} \cdot G_{\mu\nu} \cdot V
\end{equation}

\subsection{Modified Energy Adjusted for the Universe’s Size}
\begin{equation}
E_{\text{mod}} = m \cdot c^2 \cdot \left(\frac{H_{\text{obs}}}{H}\right)^2
\end{equation}

\subsection{Energy Associated with Dark Energy}
\begin{equation}
E_{\text{DE}} = 0.68 \cdot \sqrt{\frac{c^5 \cdot \hbar}{4 \pi G}} \cdot \sqrt{\frac{A_{\text{universe}} \cdot c^3}{4G \cdot \hbar}}
\end{equation}

\subsection{Energy Associated with Dark Matter}
\begin{equation}
E_{\text{DM}} = 0.27 \cdot \sqrt{\frac{c^5 \cdot \hbar}{4 \pi G}} \cdot \sqrt{\frac{A_{\text{universe}} \cdot c^3}{4G \cdot \hbar}}
\end{equation}

\subsection{Energy Associated with Black Holes/Ordinary Matter}
\begin{equation}
E_{\text{BH}} = 0.05 \cdot \sqrt{\frac{c^5 \cdot \hbar}{4 \pi G}} \cdot \sqrt{\frac{A_{\text{universe}} \cdot c^3}{4G \cdot \hbar}}
\end{equation}

\subsection{Number of Microstates Related to the Area of the Observable Universe}
\begin{equation}
\ln(N_{\text{universe}}) = \frac{A_{\text{universe}} \cdot c^3}{4G \cdot \hbar}
\end{equation}

\subsection{Area of the Observable Universe}
\begin{equation}
A_{\text{universe}} = 4 \pi \cdot H_{\text{universe}}^2
\end{equation}

\subsection{Entropy for the Observable Universe (Bekenstein-Hawking Entropy)}
\begin{equation}
S_{\text{BH}} = \frac{k_B \cdot A_{\text{universe}} \cdot c^3}{4 \hbar \cdot G}
\end{equation}

\subsection{Alpha Expression}
\begin{equation}
\alpha = A_{\text{universe}}
\end{equation}

\subsection{Shrinking of the Universe Given an Energy Density Tied to N}

Original Horizon of the Universe:
\begin{equation}
H_{\text{universe}} = \sqrt{\frac{3}{8 \pi G \rho}}
\end{equation}

New Energy Density After Change:
\begin{equation}
\rho_{\text{new}} = \sqrt{\frac{c^5 \cdot \hbar}{4 \pi G}} \cdot \frac{\sqrt{\ln(N_{\text{new}})}}{V_{\text{new}}}
\end{equation}

New Horizon After Shrinkage:
\begin{equation}
H_{\text{shrink}} = \sqrt{\frac{3}{8 \pi G \rho_{\text{new}}}}
\end{equation}

\subsection{Curvature Converted to Our Universe Standard}
\begin{equation}
G_{\mu\nu_{\text{standard}}} = -\frac{8 \pi G \cdot U_{\text{Appleby}}}{m \cdot c^2 \cdot \left(\frac{H_{\text{obs}}}{H}\right)^2 \cdot V}
\end{equation}

\subsection{Dark Matter as Curvature}

Curvature Tensor Related to Dark Matter:
\begin{equation}
G_{\text{DM}} = \frac{\rho_{\text{DM}} \cdot S_{\text{BH}}}{c_{\text{eff}}^2} \cdot \frac{1}{r^2}
\end{equation}

Where:
\begin{itemize}
\item \(G_{\mu\nu}\): Curvature tensor in Appleby’s model
\item \(\rho\): Energy density
\item \(c\): Speed of light
\item \(c_{\text{eff}}\): Effective speed of light, scaled by the ratio of horizon sizes
\item \(U_{\text{Appleby}}\): Gravitational potential energy in Appleby’s model
\item \(V\): Volume associated with the region of interest
\item \(H_{\text{obs}}\): Observer’s horizon
\item \(\Delta_H\): Differential horizon
\item \(H\): Total horizon of the universe
\item \(H_{\text{universe}}\): Horizon radius of the observable universe
\item \(E_{\text{mod}}\): Modified energy considering the universe’s size
\item \(m\): Mass associated with the energy
\item \(E_{\text{DE}}\): Energy associated with dark energy
\item \(E_{\text{DM}}\): Energy associated with dark matter
\item \(E_{\text{BH}}\): Energy associated with black holes/ordinary matter
\item \(N_{\text{universe}}\): Number of microstates related to the area of the observable universe
\item \(A_{\text{universe}}\): Area of the observable universe’s horizon
\item \(S_{\text{BH}}\): Bekenstein-Hawking entropy related to the horizon area of the observable universe
\item \(\alpha\): Proportionality constant scaling the entropy term, here tied to \(A_{\text{universe}}\)
\item \(\rho_{\text{new}}\): New energy density after a change tied to \(N_{\text{new}}\)
\item \(H_{\text{shrink}}\): New horizon size after shrinkage due to increased energy density
\item \(G_{\mu\nu_{\text{standard}}}\): Curvature tensor converted to the standard energy scale of our universe
\item \(G_{\text{DM}}\): Curvature tensor associated with dark matter in Appleby’s model
\item \(\rho_{\text{DM}}\): Dark matter energy density
\item \(r\): Distance from the center of mass (e.g., galactic center)
\end{itemize}

\subsection{Horizon Area for Black Hole}
\begin{equation}
A = 16\pi \cdot \left(\frac{GM}{c^2}\right)^2
\end{equation}

\subsection{Horizon Radius from Area}
\begin{equation}
R_{\text{horizon}} = \sqrt{\frac{A}{4 \pi}}
\end{equation}

\subsection{Horizon Radius in Terms of Entropy}
\begin{equation}
R_{\text{horizon}} = \sqrt{\frac{\hbar \cdot G \cdot S}{\pi \cdot k_B}}
\end{equation}

\subsection{Effective Speed of Light and Correlation Range}
\begin{equation}
R_{\text{corr}} = R_{\text{horizon}} \cdot \frac{c_{\text{eff}}}{c}
\end{equation}

Substituting \( R_{\text{horizon}} \) in \( R_{\text{corr}} \):

\begin{equation}
R_{\text{corr}} = \sqrt{\frac{\hbar \cdot G \cdot S}{\pi \cdot k_B}} \cdot \frac{H_{\text{obs}}}{H}
\end{equation}

Final Universal Correlation Range Formula:

\begin{equation}
R_{\text{corr}} = \sqrt{\frac{\hbar \cdot G \cdot S}{\pi \cdot k_B}} \cdot \frac{H_{\text{obs}}}{H}
\end{equation}

\end{document}
